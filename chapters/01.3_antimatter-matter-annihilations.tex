\subsection{Matter and antimatter in the universe}


\subsubsection{Origin of hadronic matter}

\subsubsection{A matter dominated universe: antimatter-matter asymmetry}
Our universe is to the best of our knowledge entirely dominated by matter over antimatter. In fact, by our best estimates only X\% of the baryonic mass in our universe is made of antimatter. This observation is staggering, because in all the reactions we can observe in particle physics experiments near earth, whenever new matter is produced the same amount of antimatter is produced also\cite{}. So the a priori assumption was that the universe houses as much antimatter as it does matter. And at first glance, this doesn't seem to impose any impossible constraints, as from a distance matter and antimatter are indistinguishable\footnote{This means to say that matter atoms would produce the same spectral lines as antimatter atoms, and undergo the same fusion reactions we see in stars.}. So while our solar system might be made of matter, what is to keep other solar systems, or even other galaxies from being made of antimatter? The issue arises when we look at the surroundings of solar systems or galaxies. Interstellar/intergalactic space is not completely empty, but populated at very low densities by protons and helium-4 from surrounding stars/galaxies. We know the density of protons in these regions to be about $n_\mathrm{H} \approx 1$ cm$^3$ for interstellar space\cite{}, and $n_\mathrm{H} \approx 1$ m$^3$ for intergalactic space \cite{}. And when a matter dominated region and an antimatter dominated region are next to each other, then in this vast space of low density matter, plenty of annihilations would occur. These annihilations would produce distinctive signals in gamma ray searches\cite{}, due to high energy photons emitted from annihilations. The lack of such signals\cite{}, places stringent limits on any large areas of antimatter within the observable universe, and leads us to believe that our universe is indeed dominated by matter. The source of thus matter-antimatter asymmetry is one of the big remaining mysteries of physics.  \\

It isn't known exactly how the different populations of matter and antimatter came to be. Perhaps only a minute difference between the two caused a tiny fraction more matter to be produced than antimatter. And since the majority of both annihilated, what we see today might by this tiny leftover fraction. For this reason, searches for differences between matter particles and their antimatter counterparts are looking to find even the tiniest discrepancy between the two\cite{}. 

\subsubsection{Sakharov condition}\label{sec:IntroSakharovContidion}
For an excess of baryons from an initially symmetric value, the following conditions must hold true:
\begin{itemize}
    \item Some interactions of elementary particles must violate baryon number conservation, since the net baryon number of the universe must change over time
    \item C and CP must be violate in order that there is no equality in the forward and backward rates of the baryon number violating processes. 
    \item The net flux must created in out-of-equilibrium conditions, since otherwhise CPT would assure compensation of the effect. 
\end{itemize}

The first condition is trivial. The second can be gaged from a simple example: if CP symmetry holds, then the forward and backwad reactions of the baryon number violating processes will cancel out. Elaborate. The third condition requires some more explanation. It is based on the fact that we believe that CPT symmetry is exact. Therefore, there must be a process which only happens in one direction in time. This cannot occur in an equilibrium condition, since then all reactions occur forward and backward in time. Therefore, it must be a reaction linked to out-of-equilibrium processes. 
\subsubsection{CP violations}
Talk about CP violations, in general terms with the example of the CKM phases, and then some experimental evidence (Kaon and beauty physics). 

\subsubsection{Baryogenesis within the standard model}
Talk about :
 - the higgs mechanism, that particle masses are determined by their yukawa coupling times the higgs vacuum expectation value
 - The fact that the vacuum expectation value of the higgs changes at super high temperatures
 - if you assume a first order phase transition in the early universe as it cools, there would be a phase boundary across which quark masses change
 - if the reflection/transmission coefficients of quarks/antiquarks are different, you can have a net baryon flux through the phase transition, this can be caused by CP violation in the electroweak force
 - the antibaryon excess on the other side would be removed by sphalerons (could it not also simply cause highly contracted antimatter regions? )
 - this leaves a net baryon number

\subsection{Antimatter-matter annihilations}
The lightest quarks -- $u$ and $d$ -- make up normal nuclear matter, i.e. protons $uud$ and neutrons $udd$, which are the two lightest baryons with masses of 938 MeV/$c^2$ and 939 MeV/$c^2$, respectively. Since the proton is the lightest baryon, and the baryon number must be conserved, any reaction of the proton with other matter must leave an intact proton at the end, thus never making the energy stored in the proton's mass available to create new particles. When baryons interact with their antibaryons, they annihilate, releasing their entire mass as available energy to create new particles. This is because by definition, the total baryon number of such a reaction is 0. The same is true for the annihilations of leptons and lepton number conservation, and for the conservation of electron charge in the annihilations of leptons and baryons. In principle, if a quantum number is antisymmetric under the CPT symmetry, it will be conserved by construction in antimatter-matter annihilation events and thus will never limit the available phase space of reactions. \\

But what actually happens when an antiparticle-particle pair meets and annihilates? Let us consider the case where only one quark and one antiquark annihilate, and the rest spectate. 

\subsubsection{Annihilation of $q\bar{q}$ and $l\bar{l}$ pairs}
It is simplest to start with the Feynman diagrams for the annihilations of elementary quarks and leptons. A selection of the lowest order is given in figure \ref{fig:annihilationsFeynmanElementary}. The relative contribution is proportional to the force's interaction strength to the power of the number of vertices, so $\alpha$ for the electromagnetic force, $\alpha_w$ for the weak force and $\alpha_s$ for the strong force. The three parameters have an ordering $\alpha_s$>>$\alpha$>>$\alpha_w$. Essentially, quark and leptons can annihilate with their antiparticles through electromagnetic and weak channels, which can also convert from quarks to leptons and vice versa. Quarks can additionally annihilate via a gluon into either another quark-antiquark pair of into hadron jets. For quarks, annihilation through the strong force should outweigh annihilation through the electromagnetic force by a factor $\alpha_s^2/\alpha^2 >>1$, which means that the strong channel should dominate. 

\begin{figure}
    \centering
    \includegraphics{}
    \caption{A selection of the lowest order Feynman diagrams showing the annihilations of elementary particles. Top row: Bottom row: }
    \label{fig:annihilationsFeynmanElementary}
\end{figure}

\subsubsection{Antiproton-proton annihilations}
It is important to note at the start of this chapter that there is currently no theory or even model which can describe the available data for antiproton-proton annihilations, or offer up an explanation for the underlying mechanism\cite{}. This is in stark contrast to quark-antiquark annihilation, which is just a first order QCD process. In this section we shall attempt to give an overview of the difficulties in describing this process, and thereby offer up a qualitative picture of the possible annihilation mechanisms.\\

It is then tempting to assume that in order to scale up an annihilation event, one might just be able to scale up the single Feynman diagram for quark-antiquark annihilation in order to get a description for antiproton-proton annihilation. However, the picutre is far more complicated. This can be intuitively understood by the fact that (anti)protons are made up of 3 valence (anti)quarks, but in the annihilation of (anti)proton pair, some of their valence (anti)quarks may well survive. In fact, consider the following reaction $\matrm{p\bar{p}} \rightarrow 3 \mathrm{M}$, where M denotes a meson. This can be done by simply rearranging the quark content of the proton and antiproton, which is illustrated in figure \ref{fig:Quark_Rearrangement}. Such a rearranging of the quarks can happen if the quarks can feel each others strong potential, which can be mediated through pion exchange. This effectively allows the potential for rearranging to the felt at further distances than the potential for annihilation. The annihilation potential between an antiproton-proton pair therefore can have a long range ( $\approxgt 1$ fm ) and a short range ($\approxlt 1$ fm) term, where the long range term is dominated by rearranging of quarks and antiquarks into mesons, and the short range term is dominated by quark-antiquark annihilation. The common notation of these processes is $An$ and $Rn$ for annihilation ($A$) and rearrangement ($R$) into $n$ mesons.\\

\begin{figure}
    \centering
    \includegraphics{}
    \caption{Schematic of $\matrm{p\bar{p}}$ annihilation into 3 mesons, done by rearranging the intermediate quarks but without annihilating any quark-antiquark pair.}
    \label{fig:Quark_Rearrangement}
\end{figure}

%This is further complicated by the fact that due to the release of significant energy, multiple further mesons may be formed by strong fragmentation in either of these reactions. 
One important observable to distinguish between these two different annihilation mechanisms is the production of strangeness, i.e. by the reaction $\matrm{p\bar{p}} \rightarrow 2\matrm{K}+ X\mathrm{M}$. This reaction cannot occur with a simple rearrangement of quarks\footnote{Neglecting quark-antiquark creation by string fragmentation.}, as a new $s\bar{s}$ pair has to be created. If antiproton-proton annihilation would be dominated by the rearrangement of quarks, we would expect to see almost no produced kaons, while if the quark annihilation channel would dominate, we would expect to produce Kaons almost as much as pions. In fact we observe about 5\% of final states which include kaons, suggesting that the truth lies somewhere in between the two models. \\

talk about:
- The isospin dependence of annihilation, and the constraining results
- The spin dependence of annihilation, and the constraining results

Given these considerations, the antiproton-proton annihilation cannot easily be described by pertubative QCD, and we are still missing an effective model capable of explaining the data. This is what makes an effective model of this interaction so difficult. Instead, an empirical parameterization is commonly used to describe the antiproton-proton inelastic cross section. A description accurately fitting the available data has been proposed by Tal et al. \cite{}, and is reproduced in equation \ref{eq:Talpbarpxs}. Figure \ref{fig:pbar_p_xs_data_comp} shows this parameterization and others on the available data.  

\begin{equation}\label{eq:Talpbarpxs}
    
\end{equation}

\begin{figure}
    \centering
    \includegraphics{}
    \caption{A comparison of antiproton-proton inelastic cross section parameterizations with the available data.}
    \label{fig:pbar_p_xs_data_comp}
\end{figure}
An overview of available data on the antiproton-proton annihilation data is given in table \ref{tab:pbarp_ann_data}, with the relevant observables. 

\begin{table}[]
    \centering
    \begin{tabular}{|c|c|c|}
         &  \\
         & 
    \end{tabular}
    \caption{Caption}
    \label{tab:pbarp_ann_data}
\end{table}

\subsubsection{Antiproton-nucleus annihilation}
In the previous section it has been established that while the antiproton-proton inelastic cross section has been well measured, a theoretical description is still lacking. In this section we therefore focus on the experimental results for antiproton-matter annihilations, and how we can use them to infer something about the underlying annihilation mechanism. 

talk about:
- data
- meson absorption within the nucleus
- nuclear breakup?
- formation of hypernuclei (measurements of hypernuclei lifetimes)
- scaling with the mass number A
- effect of Coulomb?
\subsubsection{Antinuclei-matter annihilations: the Glauber model and geometric scaling}\label{sec:IntroGlauber}
Having established the details of the antiproton inelastic cross section, we can now start to consider the process of antinuclei annihilation. All the considerations made for the antiproton inelastic cross section still hold true, but additionally there is also the potential between the antinucleons to consider. \textcolor{blue}{how does this affect things? make a simple argument about the wave function overlap, i.e. multiple antinucleons annihilating at the same time or not. We can make some deductions from the dbar -> pbar +X reaction, but it is not clear if those created the pbar out of newly created quarks or if they are simply annihilated one antineutron. At high energies, where the nucleons can fully resolve each other, geometric scaling should hold assuming there is no shadowing. At low energies however, it is unclear how this would actually work. thus the discussion leads into the glauber model and its low energy limitations.}

\subsubsection{The effect of the coulomb interaction on antinuclei-matter annihilations}
So far, only the effect of the strong force on the cross sections of antinuclei-nuclei annihilations has been considered, but there is also an electromagnetic component, since both particles are oppositely charged. 
