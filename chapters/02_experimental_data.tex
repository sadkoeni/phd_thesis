\section{Experimental data and experimental method}

\subsection{ALICE}
\subsubsection{Overview}
\begin{itemize}
    \item Trigger System
    \item ITS
    \item TPC
    \item TOF
	\item V0
\end{itemize}

\subsubsection{Basics of ALICE data structure}
ALICE data is split by runs, then by events, and within the event by tracks, as is shown in figure \ref{fig:ALICE_data_schematic}. Runs are periods of time during which collisions with the same conditions occurred, which means that the data taking is started and kept up until there is some problem which requires the run to be ended. This means that runs are of arbitrary length. They are denoted by LHC + the last two digits of the year in which the data was taken + a letter of the alphabeth, sequentially used, e.g.: LHC17a is the first data run taken during 2017. Once the raw data is taken, the Data Preparation Group (DPG) is responsible for doing a reconstruction pass over the data, which means to build the tracks from the individual detector hits, correcting for any calibration or distortion effects. The data structure  one is left with is a list of events, each of which contain a list of tracks. This is what is subsequently used by analysers. 

\begin{figure}
	\includegraphics[]{}
	\centering
	\caption{Schematic of the data structures within ALICE. The data is split by run periods, then by event, and within the event by tracks. }
		\label{fig:ALICE_data_schematic}
\end{figure}
\subsubsection{Collision system and event cuts}
The data provided in ALICE by necessity includes a large range of particles and impurities, since for each analysis what qualifies as impurities changes. Therefore, cuts are applied at the analysis level, to provide a much cleaner environment for the actual analysis. Within the analysis, these cuts happen on both an event and a track level, leaving a subset of tracks which can be analyzed. The goals of these cuts are: i) to cut bad quality tracks, where the PID is not certain, ii) to cut tracks of uninteresting particles for the specific analysis, e.g. weak decays in the analyses in this thesis and iii) to reduce the background, such as from secondary particles. These cuts also vary between collision systems, which is necessitated by their different properties. To exemplify, lets compare high multiplicity pp and Pb--Pb collisions. In HM pp collisions the mean multiplicity is 34, while in central lead lead collisions it is about 1000. This means that the mean occupancy of the detectors is much greater in Pb--Pb collisions, which in turn means that the tracking algorythm has a higher chance to assign a wrong cluster to a track. \\


\subsubsection{Reconstruction of raw (anti)nuclei}
\subsubsection{Raw \ratio\ \ ratio}
\subsubsection{Raw antitriton-to-triton ratio}
\subsubsection{Correction for secondaries from material spallation}
\subsubsection{Annihilations within the detector}


\subsection{Extracting the inelastic cross section from the antimatter-to-matter ratio}
The idea behind using the antimatter-to-matter ratio as the observable to measure the antinuclei inelastic cross section, is that antinuclei will annihilate in the detector material, and therefore disappear from our measurement. In order to quantify the inelastic cross section we thus need to know how many particles were originally produced, i.e. we need to normalise the antinuclei spectrum to the number of originally produced antinuclei. However, we cannot use theoretical predictions tuned to this data, since that would be a circular argument, i.e. we would get out the same inelastic cross section as we put in. Therefore, the matter nuclei are used as a proxy instead. This works very well for a few reasons. First, the matter inelastic cross section can be easily measured, and have been measured for deuterons \cite{}, helium-3\cite{} and tritons\cite{}. Second, other effects affecting acceptance or efficiency will largely cancel between the nuclei and antinuclei counterparts, since the two only differ in their charge sign. Third and perhaps most important, is the fact that at LHC energies, the baryochemical potential is very close to 0, and has been accurately measured for antiprotons. This means that we know to a very high degree of accuracy how many antinuclei are produced relative to the produced nuclei, and the other processes by which both might be lost within the detector are also well understood. Thus, the antimatter-to-matter ratio is sensitive to the antinuclei inelastic cross section, and other variables it is sensitive to are well understood and under control. This makes this ratio such a promising probe to measure the inelastic cross section.\\

Having established that the antimatter-to-matter ratio is sensitive to the inelastic cross section, it is still not trivial to extract the inelastic cross section from this observable. This difficulty is due to having to account for many processes. One example is the path which the particles take through the detector. In the magnetic field, (anti)nuclei travel on curved tracks, so the amount of matter they interact with will depend on their initial trajectory. This thus needs to be averaged over the eta distribution of the antinuclei. This is just one of many similar effects which make an analytical relationship between the antimatter-to-matter ratio and the antinuclei inelastic cross section difficult to achieve. Thankfully, there is a superior option with detailed Monte Carlo simulations. Detailed simulations of the ALICE detector using Geant4 account for such processes, and by changing the inelastic cross section in these simulations, we can probe its relationship to the antinuclei-to-nuclei ratio.
\subsubsection{Comparison of ratios with Monte Carlo simulations}\label{sec:MCSim}
In order to fairly compare the Monte Carlo simulations to the produced data, it is vital to account for the baryochemical potential\footnote{In other words: how much more antimatter particles we have for each matter particle. Given that we collide purely matter particles, there is a penalty for producing antimatter, even though at such high energies it is vanishingly small.} at such high energies. The relevant ratio of antiprotons to protons is shown in figure \ref{fig:BaryochemicalPotential}. Based on the same arguments as the formula for the coalescence parameter \ref{eq:CoalescenceParameter}, the effect on the ratio of antinuclei will be the same as to the antiproton-to-proton ratio taken to the exponent of the mass number of the antinucleus. 

\begin{figure}
    \centering
    \includegraphics{}
    \caption{Ratio of antiprotons to protons produced at mid-rapidity as a function of beam rapidity. At ALICE energies the value approaches unity, demonstrating that at such high energies antimatter and matter are produced in almost equal amounts.}
    \label{fig:BaryochemicalPotential}
\end{figure}
\subsubsection{Ratios as a function of the inelastic cross section}
\subsubsection{Non-linear error propagation}
\subsubsection{Accounting for energy losses between the primary vertex and the point of annihilation}
\subsubsection{Uncertainty coming from the material budget}
\subsubsection{Evaluating the average material for antinuclei annihilations in the ALICE detector}
\subsubsection{Independence of collision system}
The antimatter-to-matter ratio method's dependence on collision system has been investigated by redoing the analysis performed in pPb collisions in \cite{dbarIvan} for high multiplicity pp collisions. The dependence on the collision system is due to the multiplicity differences, and the resulting difference in the baryochemical potential as discussed in section \ref{sec:MCSim}. By taking the antiproton-to-proton ratio for the different collision systems and comparing them, the predicted difference between the antideuteron-to-deuteron ratio was obtained. The results are shown in figure \ref{fig:pp_pPb_dbardRatio}, which show that the differences between collisions systems are consistent with the expected deviation. This independence of the collision system is expected, since the inelastic cross section is completely independent on the collision system. This becomes especially self-evident when considering that the annihilations do not occur in the initial collisions, but rather as the antiparticles travel through the detector material.

\begin{figure}[h]
    \centering
    \includegraphics{}
    \includegraphics{}
    \caption{Ratio of the antiproton-to-proton ratios (left) and antideuteron-to-deuteron ratios (right) obtained in high multiplicity pp collisions and in pPb collisions, compared to the expected difference from the different baryochemical potentials (dashed red line).}
    \label{fig:pp_pPb_dbardRatio}
\end{figure}

