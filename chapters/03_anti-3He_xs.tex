\section{Measurement of the \ahe\ inelastic cross section}\label{sec:ResHe3SigmaInel}
\subsection{Physics motivation and overview of the analysis method}
\subsection{Analysis techniques}
\subsubsection{Antimatter-to-matter ratio method}

\subsubsection{TOF-TPC matching method}
The second method for measuring \sigmainel\ uses the fact that \ahe\ can be clearly identified in the TPC, and then those tracks can be checked again in the TOF. This method works akin to a fixed target experiment, in that a "beam" is identified by measuring \ahe\ in the TPC, this beam is then fired upon the "target", which in this case is the space frame and the TRD. Some of the nuclei will annihilate, while the others who make it through will generate a matching TOF hit, thus enabling a measurement of how much of the beam survives. \\
The advantage of this method is that only the antiparticles are required; no specific assumptions about the antimatter-to-matter ratio need to be assumed and tested. This also means that secondary correction need not be applied, since the origin of the \ahe\ has no impact on the result\footnote{Also, there are no secondary \ahe\ nuclei from material spallation, so the secondary correction is even less important.}. The disadvantage is that the acceptance of the TOF detector limits the applicability of this method to higher momenta, so it is more difficult to measure the low energy rise of the inelastic cross section. \\

The measurement of \sigmainel\ using the TPC-TOF matching method is thus complementary to the antiparticle-to-particle method described above. This analysis was not carried out as part of this work, but ties in closely with the results shown both in this chapter and in chapter \ref{sec:AntinucleiInTheCosmos}, and is thus described here. The measurement is also shown together with the measurement using the antimatter-to-matter inelastic cross section. More details about the analysis can be found in \cite{PavelAN, antiHe3XS}.

\subsection{Secondary correction}
\subsection{Results}

