







\section{Measurement of the \ahe\ inelastic cross section}\label{sec:ResHe3SigmaInel}
The measurement of the \ahe\ inelastic cross section is the most fundamental result of this thesis. This is the first measurement of this inelastic cross section, which is important not just for nuclear physics, but also for astrophysical searches for physics beyond the standard model, as discussed in section \ref{sec:Intro:AntinucleiGoldenChannel}. Historically, inelastic cross section measurements were done using fixed target experiments: a beam of the particle of interest was isolated, and then fired on a material target. By measuring the abundance of the particle before and after the target, the cross section could be measured. The difficulty in doing this for antinuclei lies in the production and isolation of an antinuclei beam, since antinuclei production is so rare and has a high $\sqrt{s}$ threshold. Even at the places where antinuclei are produced (at the LHC and at high energy heavy-ion facilities\cite{}), further isolating a beam of such particles is not feasible. In fact, out of all antinuclei ($A$>2), this method has only been applied to high energy antideuterons \cite{}, at the CERN proton synchrotron in the 70s, for particles at very high momenta of 13 GeV/$c$ and 25 GeV/$c$. Recently, roughly half a century later, the new measurement technique using the antiparticle-to-particle ratio has been shown to be able to measure the antideuteron inelastic cross section down to 500 MeV/$c$ \cite{antideuteronXS}. This measurement has now been expanded to \ahe\ \cite{antiHe3XS} and \atrit\ , and a separate complementary method (TPC/TOF method) has enabled the use of the high statistics Pb--Pb data to boost the measurements' precision. Both these new methods rely on quantifying the absorption of antinuclei as they travel through the detector material, rather than a dedicated target. The disadvantage of this approach is that the detector material is optimised to have little material budget as possible, as to not interfere with the particles of interest. Nevertheless, these methods allow us to further our knowledge of antinuclei inelastic cross sections for the first time in half a century.
\subsection{Physics motivation and overview of the analysis method}
\subsection{Analysis techniques}
\subsubsection{Antimatter-to-matter ratio method}

\subsubsection{TOF-TPC matching method}
The second method for measuring \sigmainel\ uses the fact that \ahe\ can be clearly identified in the TPC, and then those tracks can be checked again in the TOF. This method works akin to a fixed target experiment, in that a "beam" is identified by measuring \ahe\ in the TPC, this beam is then fired upon the "target", which in this case is the space frame and the TRD. Some of the nuclei will annihilate, while the others who make it through will generate a matching TOF hit, thus enabling a measurement of how much of the beam survives. \\
The advantage of this method is that only the antiparticles are required; no specific assumptions about the antimatter-to-matter ratio need to be assumed and tested. This also means that secondary correction need not be applied, since the origin of the \ahe\ has no impact on the result\footnote{Also, there are no secondary \ahe\ nuclei from material spallation, so the secondary correction is even less important.}. The disadvantage is that the acceptance of the TOF detector limits the applicability of this method to higher momenta, so it is more difficult to measure the low energy rise of the inelastic cross section. \\

The measurement of \sigmainel\ using the TPC-TOF matching method is thus complementary to the antiparticle-to-particle method described above. This analysis was not carried out as part of this work, but ties in closely with the results shown both in this chapter and in chapter \ref{sec:AntinucleiInTheCosmos}, and is thus described here. The measurement is also shown together with the measurement using the antimatter-to-matter inelastic cross section. More details about the analysis can be found in \cite{PavelAN, antiHe3XS}.

\subsection{Secondary correction}
\subsection{Results}

