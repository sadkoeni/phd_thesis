\section{Appendix}
\subsection{Current status of the evidence for and against (but mostly against) the existence of anti-stars}
As has been covered plenty already in this thesis, one of the biggest remaining mysteries in physics is the asymmetry between the amount of matter and antimatter present in our universe. It is thought that up to 95\% of luminous matter consists of matter rather than antimatter\cite{}. But how can it be known that another star is not entirely comprised of antimatter, given that the only difference in particles is their electric charge? The answer lies in the fact that even though structures within our galaxy and even universe are filled with only extremely low densities, they are not actually empty. It therefore stands to reason that if there were a region dominated by antimatter (even just a single anti-star, although a larger region seems more likely), such a region would eventually have to end and come in contact with a matter dominated region. In this volume of overlap, antimatter-matter annihilations would occur abundantly, resulting in a significant amount of high energy gamma rays. Such a specific and localized signal should be relatively easy to detect with dedicated gamma ray surveys, such as XXX\cite{} and XXX\cite{}. The lack of any evidence thereof suggests that there are not antimatter dominated regions, and thus no anti-stars. 
\subsection{Why the statistical hadronization model is not used for calculating (anti)nuclei yields from WIMP dark matter annihilations}\label{App:statHadronModel}
\subsection{Adding rotational velocities}