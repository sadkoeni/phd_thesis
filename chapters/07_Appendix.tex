\section{Appendix}
\subsection{Current status of the evidence for and against (but mostly against) the existence of anti-stars}
As has been covered plenty already in this thesis, one of the biggest remaining mysteries in physics is the asymmetry between the amount of matter and antimatter present in our universe. It is thought that up to 95\% of luminous matter consists of matter rather than antimatter. But how can it be known that another star is not entirely comprised of antimatter, given that the only difference in particles is their electric charge? The answer lies in the fact that even though structures within our galaxy and even universe are filled with only extremely low densities, they are not actually empty. It therefore stands to reason that if there were a region dominated by antimatter (even just a single anti-star, although a larger region seems more likely), such a region would eventually have to end and come in contact with a matter dominated region. In this volume of overlap, antimatter-matter annihilations would occur abundantly, resulting in a significant amount of high energy gamma rays. Such a specific and localized signal should be relatively easy to detect with dedicated gamma ray surveys, such as FermiLAT \cite{FermiLAT_Point_Sources}. The lack of any evidence thereof suggests that there are no large antimatter dominated regions, and thus no anti-stars. 
\subsection{Why the statistical hadronization model is not used for calculating (anti)nuclei yields from WIMP dark matter annihilations}\label{App:statHadronModel}
The statistical hadronization model (SHM) \cite{statHadronModel_review_2009} is the idea that particles are produced in thermal equilibrium, and is able to predict the yields of particles over many order of magnitudes based on a single parameter: the temperature. However, the model does not predict the correlations in momentum space or the spectra of antinuclei. This causes several problems when attempting to use the SHM for the prediction of antinuclei from WIMP dark matter. The first is the fact that the spectra of produced antinuclei is very relevant for their propagation and for the signal to background ratio in this detection channel. Furthermore, the lack of any indication of the temperature of this process makes it impossible to predict the yields produced. 
%\subsection{Comparison of different fitting algorithms for fitting low statistic histograms}\label{sec:App:lowStatFitting}
%Fitting low statistics histogram can introduce challenges related to the treatment of empty bins, and their corresponding errors. This occurs in particular because many basic fitting algorithms are taught/implemented as a means to fit graphs, rather than histograms. Whereas an empty bin in a histogram might be ignored, the proper treatment is a datapoint at 0 with corresponding errors. \\ The errors on empty bins in a histogram are Poisson errors, and therefore nonzero for empty bins. 