\subsection{Antinuclei in the cosmos}

\subsubsection{ Why producing antinuclei is so difficult: production mechanisms of antinuclei}\label{sec:IntroProductionAntinuclei}
\begin{equation}\label{eq:CoalescenceParameter}
    B_N = E_A \frac{d^3 N_A}{dp^3_A} \left[ \left( E_{p,n} \frac{d^3 N_{p,n}}{dp^3_{p,n}} \right)^A |_{\vec{p}_p=\vec{p}_n=\vec{p}_A/A } \right]^{-1}
\end{equation}
\subsubsection{ Why to we care: antinuclei as a golden channel for new physics}\label{sec:Intro:AntinucleiGoldenChannel}
The main reason why cosmic ray antinuclei make such an interesting probe for new physics is twofold: i) the rarity of the standard model processes which produce them means that any signal does not have to contend with a copious background and ii) that there are already viable theories of new physics -- namely WIMP dark matter -- which predict a detectable antinuclei signal. This has led to the coining of cosmic ray antinuclei as a "smoking gun" for new physics. 
%write about the history of the search for antinuclei

\subsubsection{ What affects antinuclei in cosmic rays: production, propagation and annihilation}
\subsubsection{ How can antinuclei in the cosmos be detected: AMS, GAPS, cubesats}
