

\section*{Zusamenfassung}\label{sec:preface_de}



Antikerne in Kosmischen Strahlen werden seit langem als eine vielverprechende Sonde f\”ur indirekte Suchen nach WIMP dunker Materie betrachtet, da in diesen Modellen WIMPs Antikerne durch Annihilation zu erzeugen. Sie gelten als so vielversprechende Sonde, weil das erwartete Antikernsignal von dunkler Materie bei niedrigen kinetischen Energien den Hintergrund, der von anderen astrophysikalischen Quellen erwartet wird, um mehrere Gr\"oßenordnungen \"ubersteigt. Tats\"achlich wird nur eine einzige relevante Hintergrundquelle in Betracht gezogen: der Zusammenstoß von Hochenergie-Kosmischen Strahlen mit dem interstellaren Medium. Experimente der aktuellen Generation erreichen Empfindlichkeiten, die optimistische Modelle untersuchen k\"onnen, und Experimente der n\"achsten Generation werden in der Lage sein, Signale s\"amtlicher Modelle vollst\"andig aufzul\"osen, falls es existiert.\\
Um aus einem solchen Signal Information zu entschl\"usseln, m\"ussen alle Wirkungen, die auf es einwirken, verstanden werden, und die Unsicherheiten jeder dieser Wirkungen bekannt sein. Die relevanten Prozesse sind die Produktion, Ausbreitung und schliesslich die Annihilation dieser Antikerne. Auf der Erde werden Antikerne in hochenergetischen Teilchenkollisionen in Teilchenbeschleunigern produziert. Aufgrund ihrer Seltenheit k\"onnen traditionelle "fixed-target" Experimente, die zur Messung der Annihilation-Wahrscheinlichkeiten (der sogenannten inelastischen Querschnitt) von Teilchen verwendet werden, nicht f\”ur niedrigenergetische Antikerne verwendet werden. Die in dieser Dissertation vorgestellte Arbeit verwendete eine k\"urzlich entwickelte neue experimentelle Methode, um erstmals die inelastischen Querschnitte von \ahe\ und \atrit\ zu messen, und verwendete diese Messungen, um den Einfluss der Annihilation auf den erwarteten Antikernfluss in Kosmischen Strahlen zu bestimmen. Dar\”uber hinaus wurde der gleiche Verfahren zur Bewertung des Einflusses von Antikern-Inelastischen Querschnitten auf ihre Ausbreitung auch auf Antideuteronen angewendet. Im Verlauf dieser Arbeit wurden auch die Unsicherheiten bez\”uglich der Propagation und Produktion von Antinukleonen neu evaluiert.\\
Der Inhalt meiner Dissertation ist daher die erstmalige Messung des inelastischen Wirkungsqueerschnitts der $A$=3 Antikerne \ahe\ und \atrit\ , sowie die erstmalige Bestimmung der experimentellen Unsicherheiten auf die \ahe\ - und Antideuteronfl\"usse auf Grund der Annihilation mit dem interstellaren Medium.  
\newpage
\section*{Preface}\label{sec:preface_en}

Antinuclei in cosmic rays have long been considered a golden channel for indirect WIMP dark matter searches, since WIMPs are predicted to be able to annihilate to create antinuclei. They are considered such a promising probe because the expected antinuclei signal from dark matter at low kinetic energies exceeds the background expected from other astrophysical sources by sever orders of magnitude. Indeed, only a single relevant background source is considered: the collision of high energy cosmic rays with the interstellar medium. Current generation experiments are reaching sensitivities which can probe optimistic models, and next generation experiments will be able to fully resolve any such signal, if it exists. \\
In order to decode any information from such a signal, all effects acting on it must be understood, and the uncertainties on each of these effects must me known. The relevant processes are the production, propagation, and finally annihilation of these antinuclei. On earth, antinuclei are produced in high energy particle collisions at particle colliders. Due to their rarity, traditional fixed target experiments employed to measure the annihilation probabilities (called the inelastic cross section) of particles cannot be used for low energy antinuclei. The work presented in this thesis used a recently developed new experimental method to measure the inelastic cross sections of \ahe\ and \atrit\ for the first time, and used these measurements in order to infer the effect of annihilation on the expected antinuclei flux in cosmic rays. Furthermore, the same procedure for evaluating the effect of antinuclei inelastic cross sections on their propagation has been applied to antideuterons. In the course of this work, the uncertainties concerning the propagation and production of antinuclei have also been re-evaluated. \\
The work carried out as part of my PhD has thus involved measuring the measurement of the inelastic cross sections of the $A$=3 antinuclei \ahe\ and \atrit\ , as well as using them in order to determine the experimental uncertainties on \ahe\ and antideuteron fluxes due to annihilation, both for the first time. 

\newpage

\section*{Acknowledgements}
I am very happy to be able to write this section, and to thank the many people who played a part in getting me to this point. Not just this thesis, but much of my life has been a group effort, with wonderful people supporting me in different aspects. And while I hope to have helped them as much as they have helped me, it is still a pleasure to have this opportunity to specifically thank them. \\

Firstly, to the people without whom I would not be me at all. To my parents, Mami and Nico, go the biggest thanks I can give. Mami, not only did you push me to be my best all my life, but you supported me at every stop. This entire thesis is too short to list all the things I am grateful to you for, but fundamentally I want to say thank you for the unconditional love and the needed stern talks to keep things on track when you feel I lost my way. Nico, over the past 13 years you have never imposed your role as a stepdad. But every time I needed help or advice, or just wanted to chat, you were there. I learned so much from your calm way of discussing difficult topics, in particular to listen first and talk later (ok I'm still working on that one). I found a fantastic father figure in you, better than I could ask for. Thank you! And to my siblings, Flo, Elena, Tini and Prince Vince, thank you for all the great times we have together. \\

Tina -- honey -- loving you is my great joy, and has been for the past 5 years. Nobody else has listened to my ramblings about science, DnD, and everything else as intensely as you. I love all the times I learn from you, about fliG, and flagella and cryoEM, and about cross stitch, and dance, and the musical ideas behind the How to Train your Dragon theme. I love the geeky science conversations we have about the internal thermodynamics and chemistry of cheese melting, and I love planning our future together. Thank you for being there for me through everything. \\

To Benni -- my very best friend -- ours is a friendship which has changed me in many ways. You have been my best friend for almost 15 years, even though for 11 of these we have lived in different countries. You taught me very early on that I should not be so stuck in just my ways, but that other perspectives can have just as much value. Because our friendship is not build on mutual interest like most, but on the appreciation for the different interests we have, and on a foundation of the same base values. You are always there for me. Thank you for everything!\\

Now on to the people whose contribution was more directly related to my PhD. First and foremost is of course Laura, the boss. I remember the first few meetings we had, looking at plots which I had studied for the better part of the week, and you were able to pick out the critical points where something was wrong in less than a minute. It took me a long time to realise that it was not just years of experience which made this possible, but the ability to ask critical questions of the information which is presented, to really try and understand the critical points. Out of all the things I learned from you over the last few years, I think this is what will stick with me the most. \\

Next thanks go all the people in our group, but in particular to the people I worked most closely with on the research which makes up this thesis: Ivan and Laura. Ivan, you taught me pretty much all that I know about antinuclei, and were the perfect discussion partner for all aspects of the analysis. And Laura, we worked together so closely on all the implementations in Galprop, the 100s of extra checks, the antideuteron paper, etc. We worked together through many late nights and got to know and like each other much better than I think either of us expected! We were always in the same boat whenever things were difficult, and managed to make it through together. I came to appreciate your unapologetic brand of honesty, by which you immediately get to the core of any subject. And I am very grateful that we became friends. \\

Finally, I want to thank my Master thesis supervisor, Prof. Paul Dauncey. You gave me another chance after I had struggled, otherwise I might not be handing in my PhD thesis today at all. 


\newpage
